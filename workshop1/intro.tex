% ______________________________________________________________________________
%
%   1DV607 Objectorienterad Analysis och Design med UML
%   Workshop 1 --- "Domain Modeling"
%
%  Author:  Jonas Sjöberg
%           Linnaeus University
%           js224eh@student.lnu.se
%           https://github.com/jonasjberg
%
% License:  Creative Commons Attribution 4.0 International (CC BY 4.0)
%           <http://creativecommons.org/licenses/by/4.0/legalcode>
%           See LICENSE.md for additional licensing information.
% ______________________________________________________________________________


% ______________________________________________________________________________
\section{Problem Description}\label{probdesc}
The problem description \cite{1dv607:workshop1-instructions} is stated as
follows:

\begin{quote}
The yacht club ``The jolly pirate'' has started to get problems. The club has
grown to such an extent that the members have a hard time keeping track of all
meetings, and other important dates such as the club's various competitions. It
is also difficult for a member to change information in the membership
registry, for example change address or reporting that they have a newly
purchased boat. The result of this is that members sometimes are without a
reserved berth, when it became time for launch towards the spring.

The reservation of berths is operated by the club secretary and he goes through
all members’ boats and book their berth in the spring. This is a very ``tricky''
work, members owning several boats would love to have them as close together as
possible, while they want their ``usual'' place. Much fuss and mess usually
follow the announcement of this year’s places, and sometimes even ``private''
exchanges between members occurs, which strongly opposed by the municipality
for safety reasons.

The biggest problem, however, has the club’s treasurer, it is quite impossible
for the treasurer to manage all subscriptions manually via receipts that
members come in with when they paid their fees. The fee consists of a fixed
part and a variable part, where the variable component is dependent on the
number and the size of the boats. Several members probably slips through each
year without paying a fee, the result of wich is that the club will soon go
bankrupt.


\subsection{Actors}
\subsubsection{Primary Actors}
\begin{description}
  \item[Secretary]
  Wants to book berths for members in a fair and effective manner. Wants to
  minimize ``whining'' when advertising of berts and avoid that private
  exchanges occurs. Would also be able to manage the club’s calendar where
  important events and meetings are advertised.
  
  \item[Treasurer]
  Wants to manage member payments in an efficient manner. It is also important
  that the Treasurer has access to history in the payments for accounting
  purposes.

  \item[Member]
  A person who is a member of the boat club, probably has one or more boats
  registered and want a good berth for these. Want to manage their membership
  data, including boats, as well as to have a smooth overview of their
  payments.  Would also like to be able to participate in various boat club
  meetings and social activities.
\end{description}


\subsubsection{Supporting Actors}
\begin{description}
  \item[Third-party systems for credit card payment]
  A service used to handle payment via credit card. Receives a transaction ID
  and a sum, will respond with a positive or negative result.

  \item[Third-party systems for Direct Payment]
  A service used to manage the direct payment online banking. Receives a
  transaction ID and a sum, will respond with a positive or negative result.
  In case of positive results money is deducted from the payer's account 00:00
  o'clock the next day, and the transaction may be rescinded by the payer prior
  to tis alternatively, there is not enough money in the account.
  
  \item[Third party systems for SMS payment]
  A service for payment via SMS. A special message with the transaction id is
  sent by the payer. This transaction id is provided by the service and
  includes a checksum so that typing errors are minimized.
  The payment will be deducted from the payer's phone bill.
\end{description}


\subsubsection{Offstage Actors}
\begin{description}
  \item[Municipality]
  Have an interest in knowing what boat owners have what berth. This is
  important in te event of for example example accidents, thefts or similar.
  
  \item[The tax authority]
  Have an interest in that the current regulations regarding the taxation of
  income associations are followed.
\end{description}



\subsection{Use Cases}\label{use-cases}

% USE CASE #1
% ----------------------------------------------------------------------------
\subsubsection{Authenticate}\label{usecase1}

A person wants to be authenticated as a role. The person is authenticated and
assigned a role.

Precondition: The person is not already assigned a user role.

\subsubsubsection{Main scenario}
\begin{enumerate}
  \tightlist
  \item
    The person wants to be authenticated.
  \item
    The system asks for log in information.
  \item
    The person supplies a user name and password
  \item
    The system controls the username and password and authenticates the
    person as a user role (Member, Treasurer, Secretary) in the system,
    the assigned user role is presented.
\end{enumerate}

\subsubsubsection{Secondary Scenarios}
\paragraph{The combination of user name and password is wrong}
\begin{enumerate}
  \tightlist
  \item
    The system presents an error message and asks for log in information
    again.
  \item
    Go to step 3 in main scenario.
\end{enumerate}


% USE CASE #2
% ----------------------------------------------------------------------------
\subsubsection{Pay Membership Fee}\label{usecase2}

A member wants to pay their dues. The fee and payment options are presented
and Member proceeds with payment. The payment is recorded and its status is
updated, a receipt is presented.

Primary Actor: User

\subsubsubsection{Main Scenario}

\begin{enumerate}
  \tightlist
  \item
    The member wants to pay their dues.
  \item
    The system presents the calculation of membership and a total amount due.
  \item
    The member chooses to pay via credit card.
  \item
    The system transmits a transaction id and a total to third-party system
    for Credit Card Payment.
  \item
    Third Party Credit Card Payments System for process the transaction and
    reply with a positive result.
  \item
    The system updates the Member's pay status to paid and presents a receipt
    of the transaction.
\end{enumerate}

\subsubsubsection{Secondary Scenarios}
\paragraph{The member chooses to pay via invoice}
\begin{enumerate}
  \tightlist
  \item
    The system presents an invoice for printing with a unique invoice id,
    boat club account number and amount due. The amount due is the total
    membership fee plus an administrative service charge of currently 45
    swedish crowns.
  \item
    The user confirms the invoice.
  \item
    The system updates the member's payment status to invoiced and presents
    that the payment status changed.
\end{enumerate}

\paragraph{The member chooses to pay via SMS payment}
\begin{enumerate}
  \tightlist
  \item
    The system contacts the Third Party System for SMS payment and enclose
    the sum total of the transaction.
  \item
    The third-party systems for SMS payment replies with the SMS message to
    be sent to process the payment.
  \item
    The system updates member's payment status to waiting. SMS message to be
    sent is presented.
\end{enumerate}

\paragraph{The member chooses to pay via third party systems for Direct Payment}
\begin{enumerate}
  \tightlist
  \item
    The system transmits a transaction id and a total to third-party systems
    for Direct Payment.
  \item
    Third Party System for Direct Payment process the payment and reply with
    a positive result.
  \item
    The System updates the Member's pay status to paid and presents a receipt
    of the transaction.
\end{enumerate}



% USE CASE #3
% ----------------------------------------------------------------------------
\subsubsection{View Member Profile Information}\label{usecase3}

A member wants to view their member data. Membership details are presented
including registreade boats booked berths and payment history.


% USE CASE #4
% ----------------------------------------------------------------------------
\subsubsection{Register Boat}\label{usecase4}

A member wants to register a new boat and the boat's data. The boat is
registered, the membership fee is updated and a confirmation appears.

\subsubsubsection{Main Scenario}
\begin{enumerate}
  \tightlist
  \item
    A member wants to register a new boat.
  \item
    The system asks for Boat Details.
  \item
    The member input the boat's size and type (sailboat, motorsailer,
    powerboat, kayak/canoe, other) and an optional image of the boat.
  \item
    The system presents the information for the boat to be registered,
    including cost of berth.
  \item
    The member confirms the correct information.
  \item
    The System registers the boat and assigns a berth using the current rules
    for berth assigment, updates the membership fee and presents a confirmation.
\end{enumerate}

\subsubsubsection{Secondary Scenarios}
\paragraph{The boat is registered during the ``offseason'' (October 1 to December 31)}
\begin{enumerate}
  \tightlist
  \item
    System assigns no berth and the membership fee for the current year is
    unchanged.
\end{enumerate}

\paragraph{The boat is registered in pre-season (January 1 to April 1) when no berth assignment has been made yet.}
\begin{enumerate}
  \tightlist
  \item
    System assigns no berth. This is done by the Secretary before the start
    of the season.
\end{enumerate}


% USE CASE #5
% ----------------------------------------------------------------------------
\subsubsection{Remove a Boat}\label{usecase5}

A Member To remove one of their registered boats. The boat is removed
and a confirmation appears.


% USE CASE #6
% ----------------------------------------------------------------------------
\subsubsection{Change a boat}\label{usecase6}

A Member wants to change a boat's data. The boat's data is updated and a
confirmation appears.


% USE CASE #7
% ----------------------------------------------------------------------------
\subsubsection{Send Payment Reminder}\label{usecase7}

The treasurer lists and send a reminder to those members who have not
paid enough dues for a calendar year.

\subsubsubsection{Main Scenario}
\begin{enumerate}
  \tightlist
  \item
    The treasurer wants to send a reminder to those members who have not
    paid enough dues for a previous year and indicates the year
    considered.
  \item
    The system lists the members who have a debt to the boat club during
    the year.
  \item
    The treasurer choose to print an invoice for each member (see 2.3.1)
    in the invoice also includes a reminder fee of 25\% per year behind
    schedule.
  \item
    The system prints an invoice per member and records that the member
    received a payment reminder.
\end{enumerate}


% USE CASE #8
% ----------------------------------------------------------------------------
\subsubsection{Assign Berths}\label{usecase8}

The Secretary wants to assign this season berths. The system assign and
book berths in accordance with applicable rules and update member
information.

\subsubsubsection{Main Scenario}
\begin{enumerate}
  \tightlist
  \item
    The Secretary wants to book this season berths.
  \item
    The system presents a proposal on the allocation under the current
    rules, the available berths and the previous year's allocation.
  \item
    The Secretary approves the proposal.
  \item
    The system assigns moorings for members' boats according to the
    proposal.
\end{enumerate}


% USE CASE #9
% ----------------------------------------------------------------------------
\subsubsection{List Berths}\label{usecase9}

The secretary chooses to list all berths. A list of unbooked and booked
berths with information about the boat and member is presented.


% USE CASE #10
% ----------------------------------------------------------------------------
\subsubsection{Manage Calendar Events}\label{usecase10}

The Secretary wants to add, delete or change a calendar event. Boat club
calendar is updated.


% USE CASE #11
% ----------------------------------------------------------------------------
\subsubsection{List Calendar Events}\label{usecase11}

A Member wish to list calendar events in a certain time interval. A list
of calendar events are presented with a short title and start date.


% USE CASE #12
% ----------------------------------------------------------------------------
\subsubsection{View Calendar Events}\label{usecase12}

Member to view all details for a particular calendar event. Calendar
event is presented including start and end dates.


% USE CASE #13
% ----------------------------------------------------------------------------
\subsubsection{Update Payment Status}\label{usecase13}

The Treasurer would like to update the status of payments. The system
will contact third party services and check the status of payments and
update these.


% USE CASE #14
% ----------------------------------------------------------------------------
\subsubsection{Update Invoice Status}\label{usecase14}

The Treasurer wants to change a Member's payment status to paid. The
system updates the Member's payment status to paid.

\end{quote}

