% ______________________________________________________________________________
%
%   1DV607 Objectorienterad Analysis och Design med UML
%   Workshop 1 --- "Domain Modeling"
%
%  Author:  Jonas Sjöberg
%           Linnaeus University
%           js224eh@student.lnu.se
%           https://github.com/jonasjberg
%
% License:  Creative Commons Attribution 4.0 International (CC BY 4.0)
%           <http://creativecommons.org/licenses/by/4.0/legalcode>
%           See LICENSE.md for additional licensing information.
% ______________________________________________________________________________


% ______________________________________________________________________________
\section{Conclusion}
%
% Conclusion - Here you will make you recommendation based on your analysis.
%
I have really struggled with understanding this very strange method of
designing software. Everything does not belong in a class. Orienting the
programming around objects, as in \emph{Object-Oriented} Programming; just
seems plain wrong and limiting.  This kind of UML-modeling and OO-design just
does not seem very useful when implementing systems in a procedural style, with
only a handlful of classes.

The abstract model often falls apart and become a tangled mess when attempting
to model the actually interesting parts, often ``cross-cutting concerns''. 

Most interesting problems and behaviours in non-trivial systems, where one
might benefit from modeling tools to aid in design and understanding --- are
cross-cutting.

Looking ahead, I will look at techniques to apply the object-oriented modeling
approach to my somewhat mixed (procedural, functional, object-oriented)
approach to thinking about programming, computing and complex systems.
